\chapter{Uživatelská dokumentace}
\label{usage-doc}
\section{Požadavky pro vývoj a spuštění projektu}
Pro spuštění programu je nutné mít nainstalovanou pouze Javu a to konkrétně JRE, minimálně ve verzi 1.8. Žádné další speciální požadavky kladeny nejsou -- pro získání dumpu je ale vhodné mít nějaký software, který umožňuje jeho pohodlné získání, např. VisualVM.

Pro vývoj (úpravu programu, jeho kompilaci, vytvoření spustitelného souboru JAR a~vydání nové verze) je zapotřebí následující softwarová výbava:
\begin{itemize}
    \item Java JDK -- ve stejné verzi jako JRE, tj. minimálně \texttt{1.8}.
    \item Maven 3 -- pro kompilaci modulů a vydání projektu, konkrétně byla testována poslední verze v době psaní, tj. \texttt{3.6.0}.
\end{itemize}

\section{Struktura projektu}
Projekt je rozdělen do několika modulů a následující adresářové struktury:

\begin{description}
    \item[analyzer] Obsahuje zdrojové kódy vytvořené knihovny, která je využívána obslužnou aplikací.
    \item[app] Obsahuje zdrojové kódy obslužné aplikace. Je vstupním bodem programu při jeho využití jako aplikace (resp. v případě, že nástroj není využíván formou knihovny).
    \item[doc] Pro úplnost tento adresář obsahuje zdrojové soubory tohoto textu.
    \item[sandbox] Obsahuje různé soubory, které se mohou při vývoji aplikace hodit -- především potom v adresáři \textit{data}, kde se nacházejí soubory HPROF, na kterých byl nástroj otestován. Rovněž obsahuje \textit{example-app}, tedy aplikaci pro generování testovacích dat a \textit{spring-boot-example}, což je jednoduchý projekt ve frameworku Spring Boot.
    \item[test] Obsahuje skripty pro spuštění testovacích setů.
    \item[pom.xml] Soubor nástroje Maven. Definuje rodičovský modul, který definuje jako submoduly některé z výše uvedených adresářů -- konkrétně \textit{sandbox/example-app}, \textit{app} a \textit{analyzer}.
\end{description}

\section{Kompilace projektu}
Projekt je možné zkompilovat pomocí volitelného IDE, případně právě pomocí nástroje Maven a definovaných souborů \texttt{pom.xml}. Konkrétně lze celý projekt, včetně knihovny a~obslužné aplikace, zkompilovat spuštěním následujícího příkazu v rodičovském adresáři (tedy tom, který obsahuje rodičovský \texttt{pom.xml} a který obsahuje výše popsané adresáře):

\begin{lstlisting}[frame={single}]
mvn clean package
\end{lstlisting}

Tento příklad předpokládá, že je lokace instalace Mavenu, tedy adresář se spustitelným souborem Mavenu \texttt{mvn}, součástí systémové cesty \texttt{PATH}. Pokud kompilace proběhne úspěšně, výsledný spustitelný soubor JAR se bude nacházet v adresáři \texttt{app/target}.

\section{Spuštění programu a jeho parametry}
Program se ovládá pomocí příkazové řádky. Nabízí několik parametrů:

\begin{itemize}
    \item \texttt{-p}/\texttt{----path} -- Cesta k souboru HPROF, který má program analyzovat.
    \item \texttt{-l}/\texttt{----list} -- Pokud je parametr specifikován, program jako akci provede vypsání jmenných prostorů, které se ve specifikovaném souboru HPROF nacházejí.
    \item \texttt{-n}/\texttt{----namespace} -- Pokud je specifikován tento parametr společně s hodnotou, jako akce se provede analýza specifikovaného HPROF souboru. Očekává se parametr ve formátu jmenného prostoru, jehož třídy mají být analyzovány -- tedy např. \texttt{com.\-example}.
    \item \texttt{-f}/\texttt{----fields} -- Přidá do výpisu i názvy a hodnoty proměnných nalezených instancí (pouze pro \texttt{-n}).
    \item \texttt{-c}/\texttt{----csv} -- Vypíše výsledek kromě konzole i do souboru \texttt{results.csv} (pouze pro \texttt{-n}).
    \item \texttt{-h}/\texttt{----help} -- Výpis nápovědy.
\end{itemize}

Pro spuštění programu je povinný parametr \texttt{-p} a jeden z dvojice \texttt{-l} či \texttt{-n}. Pro vypsání jmenných prostorů ze souboru \texttt{priklad.hprof}, nacházejícího se ve složce se souborem JAR (\texttt{memory-analyzer.jar}), slouží tedy tento příkaz:

\begin{lstlisting}[frame={single}]
java -jar memory-analyzer.jar -p priklad.hprof -l
\end{lstlisting}

Pro analýzu stejného souboru potom příkaz následující -- za předpokladu, že výše uvedený příkaz vypsal jako jeden z výsledků \texttt{cz.zcu.kiv}.

\begin{lstlisting}[frame={single}]
java -jar memory-analyzer.jar -p priklad.hprof -n cz.zcu.kiv
\end{lstlisting}

Následující příklad provede analýzu souboru, vypíše hodnoty proměnných z nalezených tříd a navíc výsledek uloží do CSV.

\begin{lstlisting}[frame={single}]
java -jar memory-analyzer.jar -p priklad.hprof -f -c -n 
cz.zcu.kiv
\end{lstlisting}